
%\documentclass[mathserif]{beamer}
\documentclass[handout]{beamer}
%\usetheme{Goettingen}
%\usetheme{Warsaw}
\usetheme{Singapore}



%\usetheme{Frankfurt}
%\usetheme{Copenhagen}
%\usetheme{Szeged}
%\usetheme{Montpellier}
%\usetheme{CambridgeUS}
%\usecolortheme{}
%\setbeamercovered{transparent}
\usepackage[english, activeacute]{babel}
\usepackage[utf8]{inputenc}
\usepackage{amsmath, amssymb}
\usepackage{dsfont}
\usepackage{graphics}
\usepackage{cases}
\usepackage{graphicx}
\usepackage{pgf}
\usepackage{epsfig}
\usepackage{amssymb}
\usepackage{multirow}	
\usepackage{amstext}
\usepackage[ruled,vlined,lined]{algorithm2e}
\usepackage{amsmath}
\usepackage{epic}
\usepackage{epsfig}
\usepackage{fontenc}
\usepackage{framed,color}
\usepackage{palatino, url, multicol}
%\algsetup{indent=2em}
\newcommand{\factorial}{\ensuremath{\mbox{\sc Factorial}}}
\newcommand{\BIGOP}[1]{\mathop{\mathchoice%
{\raise-0.22em\hbox{\huge $#1$}}%
{\raise-0.05em\hbox{\Large $#1$}}{\hbox{\large $#1$}}{#1}}}
\newcommand{\bigtimes}{\BIGOP{\times}}
\vspace{-0.5cm}
\title{Natural Language Processing \\ Large Languade Models Usage Patterns}
\vspace{-0.5cm}
\author[Felipe Bravo Márquez]{\footnotesize
%\author{\footnotesize  
 \textcolor[rgb]{0.00,0.00,1.00}{Felipe Bravo-Marquez}} 
  
 

\date{\today}

\begin{document}
\begin{frame}
\titlepage


\end{frame}



\section{Introduction}
\begin{frame}{Introduction}
\begin{scriptsize}
\begin{itemize}
\item Since the inception of Large Language Models, various patterns of use of this technology have emerged.
\item In this talk, we will try to organize these patterns and give a general overview of them.
\end{itemize}

 \begin{figure}[h]
        	\includegraphics[scale = 0.2]{pics/Large-Language-Models.jpg}
        \end{figure}
Source: \url{https://www.masayume.it/img/masayume/Large-Language-Models.jpg}
\end{scriptsize}
\end{frame}



\begin{frame}{Recap: What is an LLM}
\begin{scriptsize}
\begin{itemize}
\item  An autoregressive language model trained with a Transformer neural network on a large corpus (hundreds of bullions of tokens) and a large parameter space (billions) to predict the next word.
\item It is usually later aligned to work as a user assistant using techniques such as Reinforcement Learning From Human Feedback  \cite{ouyang2022training} or supervised fine-tuning.
\item Some are private (access via API or web browser): Google Bard, ChatGPT, etc.
\item Others are open (model's weights can be downloaded): Llama, LLama2, Falcon, etc.
\end{itemize}
\end{scriptsize}
\end{frame}

\begin{frame}{LLMs Usage Patterns}
\begin{scriptsize}
\begin{itemize}
\item Prompting
\item Vector Databases
\item Fine-Tuning
\item Evaluation
\item Agents
\end{itemize}
\end{scriptsize}
\end{frame}


% \url{https://ai.meta.com/llama/get-started/?trk=feed_main-feed-card_reshare_feed-article-content}.

\section{Prompting}

\begin{frame}{Prompting}
\begin{scriptsize}
\begin{itemize}
\item Prompt Engineering 
\item Chain of thought Prompting
\end{itemize}
\end{scriptsize}
\end{frame}


\section{Vector Databases}

\begin{frame}{Vector Databases}
\begin{scriptsize}
\begin{itemize}
\item Idea incorporate domain-scpefific knowledge not included during training.
\item Rely on a Vector Database embed queries, retrieve relevant documents, append them into the prompt \cite{lewis2021retrievalaugmented}.

\item \url{https://www.infoworld.com/article/3709912/vector-databases-in-llms-and-search.html}
\item \url{https://learn.deeplearning.ai/vector-databases-embeddings-applications/lesson/1/introduction}
\item \url{https://stackoverflow.blog/2023/10/09/from-prototype-to-production-vector-databases-in-generative-ai-applications/}
\end{itemize}
\end{scriptsize}

    \begin{figure}[h]
        	\includegraphics[scale = 0.4]{pics/vectordatabase.png}
        \end{figure}  


\end{frame}




\section{Fine-Tuning}

\begin{frame}{Instruction Fine-Tuning}
\begin{scriptsize}
\begin{itemize}
\item Paid Fine-Tuning (GPT-4??)
\item Alpaca, Vicuna, Llama, Llama2
\item https://blog.gopenai.com/paper-review-qlora-efficient-finetuning-of-quantized-llms-a3c857cd0cca
\end{itemize}
\end{scriptsize}
\end{frame}

\begin{frame}{Datasets for Instruction Fine-Tuning}
\begin{scriptsize}
\begin{itemize}
\item Standford Alpaca Dataset (Vicuna)
\item ShareGPT (Alpaca)
\item Dolly-15K
\item Orca Dataset
\end{itemize}
\end{scriptsize}
\end{frame}

\begin{frame}{Parameter Efficient Fine Tuning}
\begin{scriptsize}
\begin{itemize}
\item Lora, QLora
\item https://blog.gopenai.com/paper-review-qlora-efficient-finetuning-of-quantized-llms-a3c857cd0cca
\end{itemize}
\end{scriptsize}
\end{frame}

\begin{frame}{Token-Incrementation}
\begin{scriptsize}
\begin{itemize}
\item Lora, QLora
\item https://blog.gopenai.com/paper-review-qlora-efficient-finetuning-of-quantized-llms-a3c857cd0cca
\end{itemize}
\end{scriptsize}
\end{frame}

\section{Evaluation}
\begin{frame}{LLMBench and LLm Arena}
\begin{scriptsize}
\begin{itemize}
\item MT-bench (categories)
\item HuggingFace Open LLM Leaderboard
\item LLM Arena
\end{itemize}
\end{scriptsize}
\end{frame}



\section{Agents}

\begin{frame}{LangChain and Agents}
%https://arxiv.org/pdf/2304.03442.pdf
%https://bootcamp.uxdesign.cc/a-comprehensive-and-hands-on-guide-to-autonomous-agents-with-gpt-b58d54724d50
\begin{scriptsize}
\begin{itemize}
\item Bla
\end{itemize}
\end{scriptsize}
\end{frame}



\begin{frame}
\frametitle{Questions?}
%\vspace{1.5cm}
\begin{center}\LARGE Thanks for your Attention!\\ \end{center}



\end{frame}

\begin{frame}[allowframebreaks]\scriptsize
\frametitle{References}
\bibliography{bio}
\bibliographystyle{apalike}
%\bibliographystyle{flexbib}
\end{frame}  


%%%%%%%%%%%%%%%%%%%%%%%%%%%

\end{document}
